\documentclass{article}
\usepackage{fp}
\begin{document}
\FPpow\temp{10}{0.5}
\temp

\FPpow\temp{\temp}{2}
\FPround\temp{\temp}{15}
\temp
\end{document}

Use fp's power function for square roots and the like. With fp's accuracy you will also minimize round off errors to near 15 decimal places.

Consider the square root of 10. You write it as a power function:

\FPpow\temp{10}{0.5}

We get 3.162277660168379312

You access the value in the varaibale \temp (it can be ny name, fp will define it on the fly).

Translating this back to a power of two we geet:

10.000000000000000

Minimal shown below:


\documentclass{article}
\usepackage{fp}
\begin{document}
\FPpow\temp{10}{0.5}
\temp

\FPpow\temp{\temp}{2}
\FPround\temp{\temp}{15}
\temp
\end{document}
