% Short MWE to demonstrate use of creating a 
% layout suitable for letterheads
% Dr Y. Lazarides
% 
\documentclass[a4paper]{article}
\usepackage{graphicx}
\graphicspath{{./img/}}
\usepackage{color}
\usepackage{xcolor}
\definecolor{grey}{HTML}{F4F4F4}
\definecolor{darkgrey}{HTML}{333333}
\usepackage[paperwidth=130mm,top=-5pt, paperheight=3.4cm,left=0.5cm,right=0.5cm]{geometry}
\parindent0pt
\def\addslogan#1{%
  \fbox{\kern2pt \sloganfonthook #1}
}
\def\sloganfonthook{\large\sffamily \color{darkgrey}}
\pagestyle{empty}
\begin{document}
\pagecolor{grey}
\fboxsep0pt
\fboxrule0pt
\fbox{\vbox to 0pt{\hbox to 12cm{\hfill \fbox{\includegraphics[height=3cm]{./graphics/amato}}}}}
\vskip30pt
\fbox{\vbox{\fbox{\Large\textcolor{darkgrey}{\textbf{OUR COMPANY Inc.}}}\par
}}
\addslogan{Everything  about typesetting with \TeX\ and friends}
\end{document}

For code that you might encounter many times, is best to write macros.



Your suggestion that the code is longer than html/css/javascript is doubtful. I understand you coming from float:right, but you forgot all about having to clear divs etc. When he actually based a lot of his research on TeX. Like learning anew language you need to spend a bit of time to develop your style and base knowledge. And the almost impossible way to vertically center elements in html/css.