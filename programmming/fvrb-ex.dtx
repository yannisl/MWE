% \iffalse meta-comment, etc.
%%
%% Package `fvrb-ex' (`fvrb-ex', `hbaw' and `hcolor')
%%
%% COPYING:
%% This package may be distributed under the terms of the LaTeX Project Public
%% License, as described in lppl.txt in the base LaTeX distribution.
%% Either version 1.0 or, at your option, any later version.
%%
%% Denis Girou (CNRS/IDRIS - France) <Denis.Girou@idris.fr> March 27, 1998
%%
% \fi
%
% \changes{v1.9}{2010/05/16}{Use LPPL as license and fix bug with loading pstricks (hv)}
% \changes{v1.8}{2010/01/04}{Use ^^ instead of eight-bits chars. [KB/ER]}
% \changes{v1.7}{1998/03/26}{First public release.}
% \changes{v0.1}{1995}{First personal version.}
%
% \newif\ifmulticols
% \IfFileExists{multicol.sty}{\multicolstrue}{\multicolsfalse}
% \newif\ifpstricks
% \IfFileExists{pstricks.sty}{\pstrickstrue}{\pstricksfalse}
%
% \DoNotIndex{\\,\^,\�,\�,\�,\�} ^^A Unefficient...
% \DoNotIndex{\@defpar,\@gobble,\@ifnextchar,\@tempdimb}
% \DoNotIndex{\active,\advance,\Answer@No,\Answer@Yes,\author}
% \DoNotIndex{\begin,\begingroup,\box}
% \DoNotIndex{\catcode,\center,\circle,\CodelineIndex,\color,\ColorVersion}
% \DoNotIndex{\date,\DeclareOption,\def,\DocInput,\documentclass,\dp}
% \DoNotIndex{\else,\EnableCrossrefs,\end,\endcenter,\endgroup,\endinput}
% \DoNotIndex{\endpspicture,\expandafter}
% \DoNotIndex{\fboxsep,\fcolorbox,\fi,\filedate,\fileversion,\footnotesize}
% \DoNotIndex{\FV@XRightMargin,\fvset,\GetFileInfo,\hbadness,\hbox,\hfuzz}
% \DoNotIndex{\iden,\IfFileExists,\ifmmode,\ifpstricks,\ifx,\input}
% \DoNotIndex{\jobname,\large}
% \DoNotIndex{\m@th,\maketitle,\mathbf,\mathit,\mathnormal}
% \DoNotIndex{\mathsl,\mathtt,\message,\mu,\multiply}
% \DoNotIndex{\NeedsTeXFormat,\newcommand,\newif,\newpsobject,\newread}
% \DoNotIndex{\Oldmakeindex,\OnlyDescription}
% \DoNotIndex{\par,\parindent,\pounds,\PrintChanges,\PrintIndex}
% \DoNotIndex{\ProcessOptions,\ProvidesPackage,\psellipse,\pspicture}
% \DoNotIndex{\PSTricksLoaded,\pstrickstrue,\put}
% \DoNotIndex{\Question@Color,\Question@Mark}
% \DoNotIndex{\read,\RecordChanges,\RecustomVerbatimEnvironment,\relax}
% \DoNotIndex{\RequirePackage}
% \DoNotIndex{\section,\setbox,\setlength,\smallskip,\space,\strip}
% \DoNotIndex{\textbf,\textcolor,\textit,\textsc,\textsf,\textsl,\texttt}
% \DoNotIndex{\textwidth,\title,\topsep,\ttyin}
% \DoNotIndex{\underline,\unitlength,\usepackage}
% \DoNotIndex{\VerbatimEnvironment,\VerbatimInput,\vspace,\z@}
%
% \setcounter{IndexColumns}{2}
%
% \newcommand{\FVrbPackage}{`\textsf{fancyvrb}'}
% \newcommand{\FVrbExPackage}{`\textsf{fvrb-ex}'}
% \newcommand{\HbawPackage}{`\textsf{hbaw}'}
% \newcommand{\HcolorPackage}{`\textsf{hcolor}'}
%
% \title{The `\textsf{fvrb-ex}' package\\
%        Example environments\\with the \FVrbPackage{} package}
% \author{Denis Girou\\CNRS/IDRIS\\Orsay -- France\\
%         {\footnotesize email: Denis.Girou@idris.fr}}
% \date{Version 1.9\\\today}
%
% \maketitle
%
% \begin{abstract}
%     This package, built above the \FVrbPackage{} one (from Timothy
%   \textsc{van Zandt}), offer several kinds of the so-called \emph{example}
%   environments to format some code both in ``verbatim'' mode and in the
%   ``normal'' way, below or on the side. The advantage of such environments
%   is that the code itself is included only one time in the source code,
%   which allow to be sure of the consistence of the two versions shown.
%
%     Some other kinds of such environments are specially devoted to graphics,
%   allowing to give the size of them. It is possible in this case to draw
%   also a grid.
% \end{abstract}
%
% \ifmulticols
%   \setlength{\columnseprule}{0.6pt}
%   \begin{multicols}{2}
%   {\parskip 0pt                ^^A We have to reset \parskip (bug in \LaTeX)
% \fi
%     \tableofcontents
% \ifmulticols
%   }
%   \end{multicols}
% \fi
%
% \section{Introduction}
%
%   These macros are based on some previous work of Timothy \textsc{van
% Zandt}, adapted and developed to suit my personal needs.
%
%   This package is built above the \FVrbPackage{} one (from Timothy
% \textsc{van Zandt}), to offer some \emph{example} environments showing both
% the code and it result. It main strength is that it allow to use all the
% power of \FVrbPackage{}, with it great number of customization parameters.
%
%   These macros can also be used in conjunction with the \HbawPackage{} and
% \HcolorPackage{} packages, to allow to generate the verbatim code with
% some \emph{highlighting} attributes to emphasize parts of the text.
% It can also produce different effects according to the choice of a
% \emph{colored} or \emph{black and white} version. This last facility was
% developed for slides, to allow to generate them both in color for
% projection and in black and white to distribute them as paper copy.
%
%   Some special environments for graphic drawings allow to define directly
% the size of them, without requiring to use also a \emph{picture}
% environment. To be able to use them, the PSTricks package must be available,
% even if these specialized environments can be used for graphics built with
% another macro language than PSTricks.
%
% \begin{quote}
%     \textbf{\large Warning!} You must be aware that it has been reported
%   that this package doesn't work at all on some platforms, due to the way
%   the 8~bits characters are managed by some \TeX{} systems.
% \end{quote}
%
% \section{User interface}
%
% \begin{quote}
%     \textbf{\large Warning!} We suppose here that you already know the
%   \FVrbPackage{} package. If not, look at it own documentation!
% \end{quote}
%
% \subsection{Environments}
%
% \noindent Five new environments are defined:
%
% \begin{description}
%   \item[Example]: show the verbatim text and the formatted result below.
%   \item[CenterExample]: same than \texttt{Example}, but the result is
%   centered.
%   \item[SideBySideExample]: show the formatted result on the left and the
%   verbatim text on the right. The result is centered vertically according to 
%   the text.
%   \item[PCenterExample]: same than \texttt{CenterExample}, but the
%   result is put inside a PSTricks \texttt{pspicture} environment. It is
%   undefined if PSTricks is not available. It is specially devoted to graphic
%   drawings, but not specially built with PSTricks itself. It require to
%   specify the dimensions of the graphic. In fact, it is the same thing than
%   to use the \texttt{CenterExample} environment and to put the material
%   inside a \LaTeX{} \texttt{picture} or PSTricks \texttt{pspicture}
%   environment, but it can be more convenient to have not to specify this
%   one explicitely.
%   \item[PSideBySideExample]: same than \texttt{SideBySideExample}, but the
%   result is put inside a PSTricks \texttt{pspicture} environment. The
%   preceding comments for \texttt{PCenterExample} are of course also valid
%   for it.
% \end{description}
%
% \noindent The syntax of the first three is:
%
% \begin{Verbatim}
%   \begin{EnvironmentName}[optional_fancyvrb_arguments]
%     ..................................................
%   \end{EnvironmentName}
% \end{Verbatim}
%
% \noindent and for the two last ones:
%
% \begin{Verbatim}
%   \begin{EnvironmentName}[opt_fvrb_args][(x_min,y_min)](x_max,y_max)
%     ................................................................
%   \end{EnvironmentName}
% \end{Verbatim}
%
%   In these last cases, default values for \verb+x_min+ and \verb+y_min+ are
% 0.
%
% \subsection{Loading options}
%
% \begin{description}
%   \item[baw] : allow highlighting for a \emph{b}lack \emph{a}nd \emph{w}hite
%   version. In this case the \HbawPackage{} will be loaded and it definitions
%   will be active to emphasize texts.
%   \item[color] : allow highlighting for a \emph{color} version. In this case
%   the \HcolorPackage{} will be loaded and it definitions will be active to
%   emphasize texts.
%   \item[bawcolor] : doesn't specify in the file if it will be a \emph{b}lack
%   \emph{a}nd \emph{w}hite or a \emph{color} version to generate. A question 
%   will be asked interactively at compile time. This allow to generate at
%   choice one of the two versions without any change in the file.
%   \item[pstricks] : require the loading of PSTricks (which of course must be 
%   available on the system) to be able to use the special environments
%   devoted to graphics (but not at all mandatory PSTricks graphics).
% \end{description}
%
%   Of course, these three keywords are mutually exclusive. If none of the
% \texttt{baw}, \texttt{color} or \texttt{bawcolor} keyword is specified,
% none of the supplementary files will be loaded.
%
% \subsection{\FVrbPackage{} options imposed}
%
% \noindent The following \FVrbPackage{} parameters are imposed:
%
% \begin{description}
%   \item[gobble=2]: each line inside these environments is supposed to be
%   indented by 2 characters. It only concern the aspect of the source code,
%   which will be more readable like that.
%   \item[numbersep=3pt]: it will be effective only if \texttt{numbers=left}
%   or \texttt{numbers=right} will be chosen.
%   \item[commentchar=W]: it is the comment character for the source text,
%   which will not be printed in the verbatim part, but executed in the
%   formatted part. So, it allow to have the example not generated by the code
%   shown, which can be surprising for readers and must be used only with
%   care in special circumstances! Character chosen is 163 (�). If it cannot
%   be used on your system or if you have it inside your verbatim text, you
%   must change it by yourself in the package file.
%   \item[commandchars=XYZ]: respectively the \emph{escape}, \emph{beginning 
%   of group} and \emph{end of group} characters, to allow escape sequences
%   (\LaTeX{} commands as font and color changes) to be applied on the
%   verbatim text, using the \HbawPackage{} or \HcolorPackage{} packages.
%   These characters are specially chosen to probably be used by nobody in
%   their codes... Characters chosen are those of codes 167, 181 and 182
%   (���). If they cannot be used on your system or if you have some of them
%   inside your verbatim text, you must made yourself the relevant changes in
%   the three files of the package.
% \end{description}
%
% \section{Usage examples}
%
% \subsection{Usage of the environments}
%
% \RecustomVerbatimEnvironment{Verbatim}{Verbatim}
%   {gobble=4,commentchar=`,numbers=left,numbersep=3pt,frame=single}
%
% \begin{Verbatim}
%   \begin{Example}
%     First verbatim line.
%     Second verbatim line.
%     Third verbatim line.
%   \end{Example}
% \end{Verbatim}
%
% \begin{Example}
% First verbatim line.
% Second verbatim line.
% Third verbatim line.
% \end{Example}
%
% \vspace{5mm}
%   It is possible to customize the verbatim environments as in the standard
% way defined by \FVrbPackage{}, locally as argument of the
% environment\footnote{Take care that you must define these parameters
% directly for the \texttt{Example}, \texttt{CenterExample} and
% \texttt{SideBySideExample} environments, but that you must put them inside
% a \cs{fvset} macro for the \texttt{PCenterExample} and
% \texttt{PSideBySideExample} ones, as in these cases you can also specify 
% some PSTricks parameters, using the \cs{psset} macro.},
% or globally using the \cs{fvset} command.
%
% \begin{Verbatim}
%   \begin{Example}[frame=lines,framerule=1mm,numbers=left]
%     First verbatim line.
%     Second verbatim line.
%     Third verbatim line.
%   \end{Example}
% \end{Verbatim}
%
% \begin{Example}[frame=lines,framerule=1mm,numbers=left]
% First verbatim line.
% Second verbatim line.
% Third verbatim line.
% \end{Example}
%
% \begin{Verbatim}
%   \begin{CenterExample}[frame=single,numbers=right]
%     First verbatim line.
%     Second verbatim line.
%     Third verbatim line.
%   \end{CenterExample}
% \end{Verbatim}
%
% \begin{CenterExample}[frame=single,numbers=right]
% First verbatim line.
% Second verbatim line.
% Third verbatim line.
% \end{CenterExample}
%
% \newlength\MyLength
% \MyLength=\textwidth
% \advance\MyLength -4.7cm
%
% \noindent
% \begin{minipage}{4.7cm}
% \begin{SideBySideExample}[xrightmargin=3cm,numbers=left]
% First
% Second
% \end{SideBySideExample}
% \end{minipage}%
% \begin{minipage}{\MyLength}
% \begin{Verbatim}
%   \begin{SideBySideExample}
%         [xrightmargin=3cm,numbers=left]
%     First
%     Second
%   \end{SideBySideExample}
% \end{Verbatim}
% \end{minipage}
%
% \ifpstricks                 ^^A If PSTricks is available
% \vspace{5mm}
%   As explained, the \texttt{PCenterExample} and \texttt{PSideBySideExample}
% environments, specially devoted to graphics, put their contents inside a
% PSTricks \texttt{pspicture} environment\footnote{The $*$ convention of the
% \texttt{pspicture} environment is not accepted here.}. So, we must define
% the size of it.
%
% \begin{Verbatim}
%   \fvset{frame=lines,framerule=0.5mm,numbers=left}
%
%   \begin{PCenterExample}(-0.5,-0.5)(0.5,0.5)
%     \setlength{\unitlength}{1cm}
%     \put(0,0){\circle{1}}
%   \end{PCenterExample}
% \end{Verbatim}
%
% \begin{PCenterExample}[frame=lines,framerule=0.5mm,numbers=left]%
%                       (-0.5,-0.5)(0.5,0.5)
% \setlength{\unitlength}{1cm}
% \put(0,0){\circle{1}}
% \end{PCenterExample}
%
% \noindent So, it is the same thing than to do:
%
% \begin{Verbatim}
%   \fvset{frame=lines,framerule=0.5mm,numbers=left}
%
%   \begin{CenterExample}
%     \setlength{\unitlength}{1cm}
%     \begin{picture}(1,1)(-0.5,-0.5)
%       \put(0,0){\circle{1}}
%     \end{picture}
%   \end{CenterExample}
% \end{Verbatim}
%
% \begin{CenterExample}[frame=lines,framerule=0.5mm,numbers=left]
% \setlength{\unitlength}{1cm}
% \begin{picture}(1,1)(-0.5,-0.5)
%   \put(0,0){\circle{1}}
% \end{picture}
% \end{CenterExample}
%
%   Using the\cs{showgrid} macro, we can require to superpose the graphic
% above a grid, which can help to built it as desired. The size of the
% picture must be at least of 1 unit in this case, and the grid is rounded to
% the next greater integer.
%
% \begin{Verbatim}
%   \showgrid
%   \begin{PCenterExample}[frame=single,numbers=left](-1,-1)(1,1)
%     \setlength{\unitlength}{1cm}
%     \put(0,0){\circle{1}}
%   \end{PCenterExample}
% \end{Verbatim}
%
% {\showgrid
% \begin{PCenterExample}[frame=single,numbers=left](-1,-1)(1,1)
% \setlength{\unitlength}{1cm}
% \put(0,0){\circle{1}}
% \end{PCenterExample}
% }
%
% \begin{Verbatim}
%   \fvset{frame=single,xrightmargin=5cm}
%   \begin{PSideBySideExample}(-2,-1)(2,1)
%     \psellipse*[linecolor=yellow](2,1)
%   \end{PSideBySideExample}
%
%   \showgrid
%   \begin{PSideBySideExample}(-2,-1)(2,1)
%     \psellipse[linestyle=dashed](2,1)
%   \end{PSideBySideExample}
% \end{Verbatim}
%
% {\fvset{frame=single,xrightmargin=5cm}
% \begin{PSideBySideExample}(-2,-1)(2,1)
% \psellipse*[linecolor=yellow](2,1)
% \end{PSideBySideExample}
%
% \showgrid
% \begin{PSideBySideExample}(-2,-1)(2,1)
% \psellipse[linestyle=dashed](2,1)
% \end{PSideBySideExample}
% }
%
% \vspace{5mm}
%   The special � character defined as the comment for \FVrbPackage{} must be
% used with care, as it allow to change the code run in the formatted part
% without showing these changes in the verbatim part. So, the code shown will
% not correspond any more in this case to the one which produce the result...
% (we must take care also to do not indent these lines, otherwise we will
% change the formatting...).
%
%   Nevertheless, in very special circumstances, it allow to do special tricks.
%
% \begin{Verbatim}[commentchar=Z]
%   \begin{CenterExample}[frame=lines,framerule=0.5mm]
%     First verbatim line.
%   �\textit{%
%     Second verbatim line.
%   �}
%   �\LARGE
%     Third verbatim line.
%   \end{CenterExample}
% \end{Verbatim}
%
% \begin{CenterExample}[frame=lines,framerule=0.5mm]
% First verbatim line.
% �\textit{%
% Second verbatim line.
% �}
% �\LARGE
% Third verbatim line.
% \end{CenterExample}
%
% \else                       ^^A If PSTricks is not available
% \begin{quote}
%     \textbf{\large Warning!} The \texttt{PCenterExample} and
%   \texttt{PSideBySideExample} are not demonstrated here, because PSTricks
%   was not found on your platform.
% \end{quote}
% \fi
%
% \subsection{Usage of the \HbawPackage{} and \HcolorPackage{} packages}
%
%   If the option \texttt{baw}, \texttt{color} or \texttt{bawcolor} is chosen, 
% we can use special commands to emphasize text in the verbatim formatting.
% It allow mainly to change the font or the color of special parts of the text.
%
% \noindent Here we suppose that the package option \texttt{baw} for the
% \FVrbExPackage{} has been chosen:
%
% \begin{Verbatim}
%   \begin{CenterExample}[frame=single,numbers=right]
%     �HLa�First� verbatim line.
%     �HLb�Second� verbatim line.
%     �HLCBWz�Third� verbatim line.
%   \end{CenterExample}
% \end{Verbatim}
%
% \begin{CenterExample}[frame=single,numbers=right]
% �HLa�First� verbatim line.
% �HLb�Second� verbatim line.
% �HLCBWz�Third� verbatim line.
% \end{CenterExample}
%
% \ifpstricks                 ^^A If PSTricks is available
% \begin{Verbatim}
%   \fvset{frame=single}
%   \begin{PSideBySideExample}[xrightmargin=5.5cm](-2,-1)(2,1)
%     \psellipse[linestyle=�HLCBWz�dashed�](2,1)
%   \end{PSideBySideExample}
%
%   \begin{PSideBySideExample}[xrightmargin=4.5cm](-2,-1)(2,1)
%     \psellipse[linestyle=�HLb�dotted�](2,1)
%   \end{PSideBySideExample}
% \end{Verbatim}
%
% {\fvset{frame=single}
% \begin{PSideBySideExample}[xrightmargin=5.5cm](-2,-1)(2,1)
% \psellipse[linestyle=�HLCBWz�dashed�](2,1)
% \end{PSideBySideExample}
%
% \begin{PSideBySideExample}[xrightmargin=4.5cm](-2,-1)(2,1)
% \psellipse[linestyle=�HLb�dotted�](2,1)
% \end{PSideBySideExample}
% }
% \fi
%
% \subsection{Thanks}
%
%   I thank you Sebastian \textsc{Rahtz} \verb+<s.rahtz@elsevier.co.uk>+,
% Thomas \textsc{Siegel} \verb+<siegel@aix520.informatik.uni-leipzig.de>+ and
% Rolf \textsc{Niepraschk}\break\relax \verb+<niepraschk@ptb.de>+ for their
% tests and comments on preliminary versions of this package.
%
%
% ^^A .................... End of the documentation part ....................
%
% \section{Driver file}
%
%   The next bit of code contains the documentation driver file for \TeX{},
% i.e., the file that will produce the documentation you are currently
% reading. It will be extracted from this file by the \texttt{docstrip}
% program.
%
%    \begin{macrocode}
%<*driver>
\documentclass{ltxdoc}
\GetFileInfo{fvrb-ex.dtx}
\usepackage[baw,pstricks]{fvrb-ex}
\EnableCrossrefs
\CodelineIndex
\RecordChanges
%%\OnlyDescription                % Comment it for implementation details
%\Oldmakeindex                   % Uncomment if your MakeIndex is pre-0.9
\hbadness=7000                  % Over and under full box warnings
\begin{document}
  % To be able to use the letter "mu"
  \catcode`\^^b5=\active
  \def^^b5{$\mu$}
  % To be able to use the letter "pound"
  \catcode`\^^a3=\active
  \def^^a3{$\pounds$}
  \DocInput{fvrb-ex.dtx}
\end{document}
%</driver>
%    \end{macrocode}
%
% \section{\FVrbExPackage{} code}
%
%<*fvrb-ex>
%
% \iffalse meta-comment, etc.
%% Package `fvrb-ex'
%%
% \fi
%
% \subsection{Preambule and options management}
%
% What we need.
%    \begin{macrocode}
\NeedsTeXFormat{LaTeX2e}
%    \end{macrocode}
%
% Who we are.
%    \begin{macrocode}
\def\fileversion{1.9}
\def\filedate{2010/05/16}
\ProvidesPackage{fvrb-ex}[\filedate]
\message{`fvrb-ex' v\fileversion, \filedate\space (Denis Girou)}
%    \end{macrocode}
%
% Require PSTricks if specified (to define the \texttt{PCenterExample} and
% \texttt{PSideBySideExample} environments).
%    \begin{macrocode}
\newif\ifpstricks\pstricksfalse
\let\LoadPStricks=\relax
\DeclareOption{pstricks}{\def\LoadPStricks{\RequirePackage{pstricks}}\pstrickstrue}
%    \end{macrocode}
%
% Declaration of the explicit black and white version.
%    \begin{macrocode}
\DeclareOption{baw}{\def\ColorVersion{n}}
%    \end{macrocode}
%
% Declaration of the explicit color version.
%    \begin{macrocode}
\DeclareOption{color}{\def\ColorVersion{y}}
%    \end{macrocode}
%
% Declaration option to choose black and white or color version.
%    \begin{macrocode}
\DeclareOption{bawcolor}{\def\ColorVersion{?}}
%    \end{macrocode}
%
% Process the options.
%    \begin{macrocode}
\ProcessOptions\relax
\LoadPStricks
%    \end{macrocode}
%
% Require the \FVrbPackage{} package.
%    \begin{macrocode}
\ifpstricks\RequirePackage{pstricks}\fi
\RequirePackage{fancyvrb}
%    \end{macrocode}
%
% To ask an interactive question if necessary (code from `docstrip').
%    \begin{macrocode}
\newread\ttyin
\def\iden#1{#1}
\def\strip#1#2 \@gobble{\def #1{#2}}
\def\@defpar{\par}
\def\@gobble#1{}
\def\Ask#1#2{%
\message{#2}\read\ttyin to #1\ifx#1\@defpar\def#1{}\else
\iden{\expandafter\strip\expandafter#1#1\@gobble\@gobble} \@gobble\fi}
%    \end{macrocode}
%
% To be able to ask later to choose between color and black and white version.
% \begin{macro}{\Answer@Yes}
%    \begin{macrocode}
\def\Answer@Yes{y}
%    \end{macrocode}
% \end{macro}
%
% \begin{macro}{\Answer@No}
%    \begin{macrocode}
\def\Answer@No{n}
%    \end{macrocode}
% \end{macro}
%
% \begin{macro}{\Question@Mark}
%    \begin{macrocode}
\def\Question@Mark{?}
%    \end{macrocode}
% \end{macro}
%
% \begin{macro}{\Question@Color}
%    \begin{macrocode}
\def\Question@Color{Color version? (y=yes)}
%    \end{macrocode}
% \end{macro}
%
% For the highlighting style (color or black and white version), if defined.
%
% \begin{macro}{\Highlight@Attributes}
%    \begin{macrocode}
\def\Highlight@Attributes{}   % Default=nothing
%    \end{macrocode}
% \end{macro}
%
% \begin{macro}{\NoHighlight@Attributes}
%    \begin{macrocode}
\def\NoHighlight@Attributes{} % Default=nothing
%    \end{macrocode}
% \end{macro}
%
% Forced choice of the \emph{black and white} version.
%    \begin{macrocode}
\ifx\ColorVersion\Answer@Yes
  \RequirePackage{color}       % Standard LaTeX `color' package
  \RequirePackage{hcolor}      % Color version
\fi
%    \end{macrocode}
%
% Forced choice of the \emph{color} version.
%    \begin{macrocode}
\ifx\ColorVersion\Answer@No
  \RequirePackage{color}       % Standard LaTeX `color' package
  \RequirePackage{hbaw}        % Black and white version
\fi
%    \end{macrocode}
%
% Choice of the highlighting style (color, black and white or nothing).
%    \begin{macrocode}
\ifx\ColorVersion\Question@Mark
  \Ask\ColorVersion{^^J\Question@Color}
  \ifx\ColorVersion\Answer@Yes
    \RequirePackage{color}      % Standard LaTeX `color' package
    \RequirePackage{hcolor}     % Color version
  \else
    \RequirePackage{color}      % Standard LaTeX `color' package
    \RequirePackage{hbaw}       % Black and white version
  \fi
\fi
%    \end{macrocode}
%
% Verbatim example environments must be indented by two spaces, which should
% be ignored.
%    \begin{macrocode}
\fvset{gobble=2}
%    \end{macrocode}
%
% To decide later if the result must be surimpose on a grid (useful only if
% PSTricks is available).
%    \begin{macrocode}
\newif\ifFvrbEx@Grid
%    \end{macrocode}
%
% \subsection{The various example environments}
%
% \DescribeEnv{Example}
% \texttt{Example} is an environment to show the verbatim code and
% the result just below.
%    \begin{macrocode}
\def\Example{%
\catcode`\^^M=\active
\@ifnextchar[{\catcode`\^^M=5\Example@}{\catcode`\^^M=5\Begin@Example}}
%    \end{macrocode}
%
% \cs{endExample} is a macro for the \texttt{Example} environment to close 
% the verbatim part and to put the formatted result below.
%    \begin{macrocode}
\def\endExample{%
\end{VerbatimOut}%
\Below@Example{\input{\jobname.tmp}}}
%    \end{macrocode}
%
% \begin{macro}{\Example@}
% \cs{Example@} is an internal macro to set locally the \FVrbPackage{} options
% if needed (both for the \texttt{Example}, \texttt{CenterExample} and
% \texttt{SideBySideExample} environments).
%    \begin{macrocode}
\def\Example@[#1]{\fvset{#1}\Begin@Example}
%    \end{macrocode}
% \end{macro}
%
% \DescribeEnv{CenterExample}
% \texttt{CenterExample} is an environment to show the verbatim code and
% the result just below, inside a \texttt{center} environment.
%    \begin{macrocode}
\def\CenterExample{%
\catcode`\^^M=\active
\@ifnextchar[{\catcode`\^^M=5\Example@}{\catcode`\^^M=5\Begin@Example}}
%    \end{macrocode}
%
% \begin{macro}{\endCenterExample}
% \cs{endCenterExample} is a macro for the \texttt{CenterExample}
% environment to close the verbatim part and to put the formatted result below,
% centering it.
%    \begin{macrocode}
\def\endCenterExample{%
\end{VerbatimOut}%
\center
\Below@Example{\input{\jobname.tmp}}
\endcenter}
%    \end{macrocode}
% \end{macro}
%
% \DescribeEnv{SideBySideExample}
% \texttt{SideBySideExample} is an environment to show the verbatim code and
% the result on the left, using a \texttt{minipage} environment.
%    \begin{macrocode}
\def\SideBySideExample{%
\catcode`\^^M=\active
\@ifnextchar[{\catcode`\^^M=5\Example@}%
             {\catcode`\^^M=5\Begin@Example}}
%    \end{macrocode}
%
% \begin{macro}{\endSideBySideExample}
% \cs{endSideBySideExample} is a macro for the \texttt{SideBySideExample}
% environment to close the verbatim part and to put the formatted result on
% the left side.
%    \begin{macrocode}
\def\endSideBySideExample{%
\end{VerbatimOut}%
\SideBySide@Example{\input{\jobname.tmp}}}
%    \end{macrocode}
% \end{macro}
%
% \subsection{General macros}
%
% \begin{macro}{\Begin@Example}
% \cs{Begin@Example} is an internal macro to start an example environment.
%    \begin{macrocode}
\newcommand{\Begin@Example}{%
\parindent=0pt
\multiply\topsep by 2
\VerbatimEnvironment
\begin{VerbatimOut}[codes={\catcode`\^^a3=12\catcode`\^^a7=12\catcode`\^^b5=12%
                           \catcode`\^^b6=12}]{\jobname.tmp}}
%    \end{macrocode}
% \end{macro}
%
% \begin{macro}{\Below@Example}
% \cs{Below@Example} is an internal macro to insert the verbatim part and
% to put the formatted result just below. The possible highlighting must
% be suppressed and the comment character desactivated before to input the
% formatted part.
%    \begin{macrocode}
\newcommand{\Below@Example}[1]{%
\VerbatimInput[gobble=0,commentchar=^^a3,commandchars=^^a7^^b5^^b6,numbersep=3pt]%
              {\jobname.tmp}
\catcode`\^^a3=9\relax%
\NoHighlight@Attributes % To suppress possible highlighting
\ifFvrbEx@Grid\vspace{5pt}\fi
#1%
\ifFvrbEx@Grid\vspace{5pt}\fi
\par}
%    \end{macrocode}
% \end{macro}
%
% \begin{macro}{\SideBySide@Example}
% \cs{SideBySide@Example} is an internal macro to insert the verbatim part and
% to put the formatted result on the left side, using a \texttt{minipage}
% environment. The possible highlighting must be suppressed and the comment
% character desactivated before to input the formatted part.
%    \begin{macrocode}
\newcommand{\SideBySide@Example}[1]{%
\@tempdimb=\FV@XRightMargin
\advance\@tempdimb -5mm
\begin{minipage}[c]{\@tempdimb}
  \fvset{xrightmargin=0pt}
  \catcode`\^^a3=9\relax%
  \NoHighlight@Attributes % To suppress possible highlighting
  #1
\end{minipage}%
\@tempdimb=\textwidth
\advance\@tempdimb -\FV@XRightMargin
\advance\@tempdimb 5mm
\begin{minipage}[c]{\@tempdimb}
  \VerbatimInput[gobble=0,commentchar=^^a3,commandchars=^^a7^^b5^^b6,numbersep=3pt,
                 xleftmargin=5mm,xrightmargin=0pt]{\jobname.tmp}
\end{minipage}}
%    \end{macrocode}
% \end{macro}
%
% \subsection{Example environments using the \texttt{pspicture} PSTricks one}
%
% Of course, PSTricks must be available to be able to use them.
%    \begin{macrocode}
\ifx\PSTricksLoaded\endinput
%    \end{macrocode}
%
% Grid definition (using PSTricks).
%    \begin{macrocode}
  \newcommand{\showgrid}{\FvrbEx@Gridtrue}
  \newpsobject{FvrbExGrid}{psgrid}{subgriddiv=0,griddots=10,gridlabels=7pt}
%    \end{macrocode}
%
% \DescribeEnv{PCenterExample}
% \texttt{PCenterExample} is an environment to show the verbatim code and
% the result just below, inside a \texttt{center} environment.
%    \begin{macrocode}
  \def\PCenterExample{\@ifnextchar[{\Pst@Example}{\Pst@@Example}}
%    \end{macrocode}
%
% \begin{macro}{\endPCenterExample}
% \cs{endPCenterExample} is a macro for the \texttt{PCenterExample}
% environment to close the verbatim part and to put the formatted result below,
% inside a PSTricks \texttt{pspicture} environment, and centering it.
%    \begin{macrocode}
  \def\endPCenterExample{%
    \end{VerbatimOut}%
    \Below@Example{%
      \center
      \expandafter\pspicture\Picture@Size
      \ifFvrbEx@Grid\FvrbExGrid\fi\relax
      \input{\jobname.tmp}%
      \endpspicture
      \endcenter
      \smallskip}}
%    \end{macrocode}
% \end{macro}
%
% \DescribeEnv{PSideBySideExample}
% \texttt{PSideBySideExample} is an environment to show the verbatim code and
% to put the formatted result on the left side, inside a PSTricks
% \texttt{pspicture} environment.
%    \begin{macrocode}
  \def\PSideBySideExample{\@ifnextchar[{\Pst@Example}{\Pst@@Example}}
%    \end{macrocode}
%
% \begin{macro}{\endPSideBySideExample}
% \cs{endPSideBySideExample} is a macro for the \texttt{PSideBySideExample}
% environment to close the verbatim code and to put the formatted result on
% the left side, inside a PSTricks \texttt{pspicture} environment.
%    \begin{macrocode}
  \def\endPSideBySideExample{%
    \end{VerbatimOut}%
    \SideBySide@Example{%
      \ifFvrbEx@Grid\vspace{5pt}\fi
      \expandafter\pspicture\Picture@Size
      \ifFvrbEx@Grid\FvrbExGrid\fi\relax
      \input{\jobname.tmp}%
      \endpspicture
      \ifFvrbEx@Grid\vspace{5pt}\fi
      \smallskip}}
%    \end{macrocode}
% \end{macro}
%
% \begin{macro}{\Pst@Example}
% \cs{Pst@Example} is an internal macro to set locally the \FVrbPackage{}
% options if needed (both for \texttt{PCenterExample} and
% \texttt{PSideBySideExample} environments).
%    \begin{macrocode}
  \def\Pst@Example[#1]{\fvset{#1}\Pst@@Example}
%    \end{macrocode}
% \end{macro}
%
% \begin{macro}{\Pst@@Example}
% \cs{Pst@@Example} is an internal macro to define the starting point of the
% \texttt{pspicture} environment to used.
%    \begin{macrocode}
  \def\Pst@@Example#1(#2,#3){%
    \catcode`\^^M=\active
    \@ifnextchar({\catcode`\^^M=5\Pst@@@Example(#2,#3)}
                 {\catcode`\^^M=5\Pst@@@Example(0,0)(#2,#3)}}
%    \end{macrocode}
% \end{macro}
%
% \begin{macro}{\Pst@@@Example}
% \cs{Pst@@@Example} is an internal macro to transmit the size of the
% \texttt{pspicture} environment to used and to call the relevant internal
% macro to insert the verbatim part.
%    \begin{macrocode}
  \def\Pst@@@Example(#1,#2)(#3,#4){%
    \def\Picture@Size{(#1,#2)(#3,#4)}%
    \Begin@Example}
%    \end{macrocode}
% \end{macro}
%
% End of the code for environments using PSTricks.
%    \begin{macrocode}
\fi                             % End \ifx\PSTricksLoaded
%    \end{macrocode}
%</fvrb-ex>
%
% \section{\HbawPackage{} code}
%
%<*hbaw>
%
% \iffalse meta-comment, etc.
%% Package `hbaw'
%%
% \fi
%
% What we need.
%    \begin{macrocode}
\NeedsTeXFormat{LaTeX2e}
%    \end{macrocode}
%
% Who we are.
%    \begin{macrocode}
\def\fileversion{1.4}
\def\filedate{1998/03/19}
\ProvidesPackage{hbaw}[\filedate]
\message{`hbaw' v\fileversion, \filedate\space (Denis Girou)}
%    \end{macrocode}
%
% \begin{macro}{\FvrbEx@ColoredBox}
% \cs{FvrbEx@ColoredBox} is an internal macro to print some text in bold face 
% in a defined color, inside a colored box of another color.
%    \begin{macrocode}
\newcommand{\FvrbEx@ColoredBox}[3]{%
\fboxsep=1pt\fcolorbox{#2}{#2}{\textcolor{#3}{\textbf{#1}}}}
%    \end{macrocode}
% \end{macro}
% 
% \begin{macro}{\Highlight@Attributes}
% \cs{Highlight@Attributes} is an internal macro to define a serie of
% highlighting macros to emphasize text in a black and white mode.
% All have a corresponding version in color mode, using the \HcolorPackage{}
% package. We take care here of possible mathematic material.
%    \begin{macrocode}
\def\Highlight@Attributes{%
%    \end{macrocode}
% \end{macro}
%
% Some font changes.
%    \begin{macrocode}
\def\HLa##1{\ifmmode\mathbf{##1}\else\textbf{##1}\fi}
\def\HLb##1{\ifmmode\mathsl{##1}\else\textsl{##1}\fi}
\def\HLc##1{##1}
\def\HLd##1{##1}
\def\HLe##1{\ifmmode\mathbf{##1}\else\textbf{##1}\fi}
\def\HLf##1{##1}
\def\HLq##1{##1}
\def\HLr##1{##1}
\def\HLz##1{##1}
%    \end{macrocode}
%
% Bold text.
%    \begin{macrocode}
\def\HLBFa##1{\ifmmode\mathbf{##1}\else\textbf{##1}\fi}
\def\HLBFb##1{\ifmmode\mathbf{##1}\else\textbf{##1}\fi}
\def\HLBFc##1{\ifmmode\mathbf{##1}\else\textbf{##1}\fi}
\def\HLBFd##1{\ifmmode\mathbf{##1}\else\textbf{##1}\fi}
\def\HLBFe##1{\ifmmode\mathbf{##1}\else\textbf{##1}\fi}
\def\HLBFf##1{\ifmmode\mathbf{##1}\else\textbf{##1}\fi}
\def\HLBFz##1{\ifmmode\mathbf{##1}\else\textbf{##1}\fi}
%    \end{macrocode}
%
% Italic text (\verb+\textsl+ rather than \verb+\textit+ due to the problem
% of the coding of the \$ character).
%    \begin{macrocode}
\def\HLITa##1{\ifmmode\mathnormal{##1}\else\textsl{##1}\fi}
\def\HLITb##1{\ifmmode\mathnormal{##1}\else\textsl{##1}\fi}
\def\HLITc##1{\ifmmode\mathnormal{##1}\else\textsl{##1}\fi}
\def\HLITd##1{\ifmmode\mathnormal{##1}\else\textsl{##1}\fi}
\def\HLITe##1{\ifmmode\mathnormal{##1}\else\textsl{##1}\fi}
\def\HLITf##1{\ifmmode\mathnormal{##1}\else\textsl{##1}\fi}
\def\HLITz##1{\ifmmode\mathnormal{##1}\else\textsl{##1}\fi}
%    \end{macrocode}
%
% Small capitals text.
%    \begin{macrocode}
\def\HLSCa##1{\ifmmode\mathit{##1}\else\textsc{##1}\fi}
\def\HLSCb##1{\ifmmode\mathit{##1}\else\textsc{##1}\fi}
\def\HLSCc##1{\ifmmode\mathit{##1}\else\textsc{##1}\fi}
\def\HLSCd##1{\ifmmode\mathit{##1}\else\textsc{##1}\fi}
\def\HLSCe##1{\ifmmode\mathit{##1}\else\textsc{##1}\fi}
\def\HLSCf##1{\ifmmode\mathit{##1}\else\textsc{##1}\fi}
\def\HLSCz##1{\ifmmode\mathit{##1}\else\textsc{##1}\fi}
%    \end{macrocode}
%
% Teletype writer text.
%    \begin{macrocode}
\def\HLTTa##1{\ifmmode\mathtt{##1}\else\texttt{##1}\fi}
\def\HLTTb##1{\ifmmode\mathtt{##1}\else\texttt{##1}\fi}
\def\HLTTc##1{\ifmmode\mathtt{##1}\else\texttt{##1}\fi}
\def\HLTTd##1{\ifmmode\mathtt{##1}\else\texttt{##1}\fi}
\def\HLTTe##1{\ifmmode\mathtt{##1}\else\texttt{##1}\fi}
\def\HLTTf##1{\ifmmode\mathtt{##1}\else\texttt{##1}\fi}
\def\HLTTq##1{\ifmmode\mathtt{##1}\else\texttt{##1}\fi}
\def\HLTTr##1{\ifmmode\mathtt{##1}\else\texttt{##1}\fi}
\def\HLTTz##1{\ifmmode\mathtt{##1}\else\texttt{##1}\fi}
%    \end{macrocode}
%
% Italic and teletype writer text.
%    \begin{macrocode}
\def\HLITTTa##1{\ifmmode\mathtt{##1}\else\textsl{\texttt{##1}}\fi}
\def\HLITTTb##1{\ifmmode\mathtt{##1}\else\textsl{\texttt{##1}}\fi}
\def\HLITTTc##1{\ifmmode\mathtt{##1}\else\textsl{\texttt{##1}}\fi}
\def\HLITTTd##1{\ifmmode\mathtt{##1}\else\textsl{\texttt{##1}}\fi}
\def\HLITTTe##1{\ifmmode\mathtt{##1}\else\textsl{\texttt{##1}}\fi}
\def\HLITTTf##1{\ifmmode\mathtt{##1}\else\textsl{\texttt{##1}}\fi}
\def\HLITTTz##1{\ifmmode\mathtt{##1}\else\textsl{\texttt{##1}}\fi}
%    \end{macrocode}
%
% Black text inside a colored box.
%    \begin{macrocode}
\def\HLCBBa##1{\FvrbEx@ColoredBox{##1}{blue}{black}}
\def\HLCBBb##1{\FvrbEx@ColoredBox{##1}{cyan}{black}}
\def\HLCBBc##1{\FvrbEx@ColoredBox{##1}{green}{black}}
\def\HLCBBd##1{\FvrbEx@ColoredBox{##1}{magenta}{black}}
\def\HLCBBe##1{\FvrbEx@ColoredBox{##1}{red}{black}}
\def\HLCBBf##1{\FvrbEx@ColoredBox{##1}{yellow}{black}}
\def\HLCBBz##1{\FvrbEx@ColoredBox{##1}{black}{black}}
%    \end{macrocode}
%
% White text inside a colored box (we replace cyan and yellow by green because
% these colors are not well seen in black and white mode).
%    \begin{macrocode}
\def\HLCBWa##1{\FvrbEx@ColoredBox{##1}{blue}{white}}
\def\HLCBWb##1{\FvrbEx@ColoredBox{##1}{green}{white}}
\def\HLCBWc##1{\FvrbEx@ColoredBox{##1}{green}{white}}
\def\HLCBWd##1{\FvrbEx@ColoredBox{##1}{magenta}{white}}
\def\HLCBWe##1{\FvrbEx@ColoredBox{##1}{red}{white}}
\def\HLCBWf##1{\FvrbEx@ColoredBox{##1}{green}{white}}
\def\HLCBWz##1{\FvrbEx@ColoredBox{##1}{black}{white}}
%    \end{macrocode}
%
% Underlined text.
%    \begin{macrocode}
\def\HLSa##1{\underline{##1}}
\def\HLSb##1{\underline{##1}}
\def\HLSc##1{\underline{##1}}
\def\HLSd##1{\underline{##1}}
\def\HLSe##1{\underline{##1}}
\def\HLSf##1{\underline{##1}}
\def\HLSz##1{\underline{##1}}
%    \end{macrocode}
%
% Underlined text (same than preceding in this black and white version).
%    \begin{macrocode}
\def\HLSaa##1{\underline{##1}}
\def\HLSbb##1{\underline{##1}}
\def\HLScc##1{\underline{##1}}
\def\HLSdd##1{\underline{##1}}
\def\HLSee##1{\underline{##1}}
\def\HLSef##1{\underline{##1}}
\def\HLSez##1{\underline{##1}}
%    \end{macrocode}
%
% End of \cs{Highlight@Attributes}.
%    \begin{macrocode}
}
%    \end{macrocode}
%
% \begin{macro}{\NoHighlight@Attributes}
%     \cs{NoHighlight@Attributes} is an internal macro to inhibit all
%   the highlighting macros define by \cs{Highlight@Attributes}. It is
%   necessary to call it before to insert the formatted part, as highlighting
%   process must concern only the verbatim part.
% \end{macro}
%
%    \begin{macrocode}
\def\NoHighlight@Attributes{%
%    \end{macrocode}
%
% First, we re-establish the active catcodes for the verbatim mode.
%    \begin{macrocode}
\catcode`\^^a7=0\relax%
\catcode`\^^b5=1\relax%
\catcode`\^^b6=2\relax%
%    \end{macrocode}
%
% Desactivation of the highlighting macros.
%    \begin{macrocode}
\def\HLa##1{##1}%
\def\HLb##1{##1}%
\def\HLc##1{##1}%
\def\HLd##1{##1}%
\def\HLe##1{##1}%
\def\HLf##1{##1}%
\def\HLBFa##1{##1}%
\def\HLBFb##1{##1}%
\def\HLBFc##1{##1}%
\def\HLBFd##1{##1}%
\def\HLBFe##1{##1}%
\def\HLBFf##1{##1}%
\def\HLITa##1{##1}%
\def\HLITb##1{##1}%
\def\HLITc##1{##1}%
\def\HLITd##1{##1}%
\def\HLITe##1{##1}%
\def\HLITf##1{##1}%
\def\HLCBBa##1{##1}%
\def\HLCBBb##1{##1}%
\def\HLCBBc##1{##1}%
\def\HLCBBd##1{##1}%
\def\HLCBBe##1{##1}%
\def\HLCBBf##1{##1}%
\def\HLCBBz##1{##1}%
\def\HLCBWa##1{##1}%
\def\HLCBWb##1{##1}%
\def\HLCBWc##1{##1}%
\def\HLCBWd##1{##1}%
\def\HLCBWe##1{##1}%
\def\HLCBWf##1{##1}%
\def\HLCBWz##1{##1}%
%    \end{macrocode}
%
% End of \cs{NoHighlight@Attributes}.
%    \begin{macrocode}
}
%    \end{macrocode}
%
% Activation of the highlighting macros.
%    \begin{macrocode}
\Highlight@Attributes
%    \end{macrocode}
%
%</hbaw>
%
% \section{\HcolorPackage{} code}
%
%<*hcolor>
%
% \iffalse meta-comment, etc.
%% Package `hcolor'
%%
% \fi
%
% What we need.
%    \begin{macrocode}
\NeedsTeXFormat{LaTeX2e}
%    \end{macrocode}
%
% Who we are.
%    \begin{macrocode}
\def\fileversion{1.4}
\def\filedate{1998/03/19}
\ProvidesPackage{hcolor}[\filedate]
\message{`hcolor' v\fileversion, \filedate\space (Denis Girou)}
%    \end{macrocode}
%
% \begin{macro}{\FvrbEx@ColoredUnderline}
% \cs{FvrbEx@ColoredUnderline} is an internal macro to underline some text in
% color.
%    \begin{macrocode}
\newcommand{\FvrbEx@ColoredUnderline}[3]{%
$\setbox\z@\hbox{\begingroup#3\endgroup}%
\dp\z@\z@\m@th\color{#1}\underline{\textcolor{#2}{\box\z@}}$}
%    \end{macrocode}
% \end{macro}
% 
% \begin{macro}{\FvrbEx@ColoredBox}
% \cs{FvrbEx@ColoredBox} is an internal macro to print some text in bold face 
% in a defined color, inside a colored box of another color.
%    \begin{macrocode}
\newcommand{\FvrbEx@ColoredBox}[3]{%
\fboxsep=1pt\fcolorbox{#2}{#2}{\textcolor{#3}{\textbf{#1}}}}
%    \end{macrocode}
% \end{macro}
% 
% \begin{macro}{\Highlight@Attributes}
% \cs{Highlight@Attributes} is an internal macro to define a serie of
% highlighting macros to emphasize text in a black and white mode.
% All have a corresponding version in black and white mode, using the
% \HbawPackage{} package.  We do not take care here of possible mathematic
% material, but it can be done...
%    \begin{macrocode}
\def\Highlight@Attributes{%
%    \end{macrocode}
% \end{macro}
%
% Some font changes.
%    \begin{macrocode}
\def\HLa##1{\textcolor{blue}{##1}}
\def\HLb##1{\textcolor{cyan}{##1}}
\def\HLc##1{\textcolor{green}{##1}}
\def\HLd##1{\textcolor{magenta}{##1}}
\def\HLe##1{\textcolor{red}{##1}}
\def\HLf##1{\textcolor{yellow}{##1}}
\def\HLq##1{\textcolor{PaleGreen}{##1}}
\def\HLr##1{\textcolor{SlateBlue}{##1}}
\def\HLz##1{\textcolor{black}{##1}}
%    \end{macrocode}
%
% Colored bold text.
%    \begin{macrocode}
\def\HLBFa##1{\textcolor{blue}{\textbf{##1}}}
\def\HLBFb##1{\textcolor{cyan}{\textbf{##1}}}
\def\HLBFc##1{\textcolor{green}{\textbf{##1}}}
\def\HLBFd##1{\textcolor{magenta}{\textbf{##1}}}
\def\HLBFe##1{\textcolor{red}{\textbf{##1}}}
\def\HLBFf##1{\textcolor{yellow}{\textbf{##1}}}
\def\HLBFz##1{\textcolor{black}{\textbf{##1}}}
%    \end{macrocode}
%
% Colored italic text (\verb+\textsl+ rather than \verb+\textit+ due to the
% problem of the coding of the \$ character).
%    \begin{macrocode}
\def\HLITa##1{\textcolor{blue}{\textsl{##1}}}
\def\HLITb##1{\textcolor{cyan}{\textsl{##1}}}
\def\HLITc##1{\textcolor{green}{\textsl{##1}}}
\def\HLITd##1{\textcolor{magenta}{\textsl{##1}}}
\def\HLITe##1{\textcolor{red}{\textsl{##1}}}
\def\HLITf##1{\textcolor{yellow}{\textsl{##1}}}
\def\HLITz##1{\textcolor{black}{\textsl{##1}}}
%    \end{macrocode}
%
% Colored small capitals text.
%    \begin{macrocode}
\def\HLSCa##1{\textcolor{blue}{\textsc{##1}}}
\def\HLSCb##1{\textcolor{cyan}{\textsc{##1}}}
\def\HLSCc##1{\textcolor{green}{\textsc{##1}}}
\def\HLSCd##1{\textcolor{magenta}{\textsc{##1}}}
\def\HLSCe##1{\textcolor{red}{\textsc{##1}}}
\def\HLSCf##1{\textcolor{yellow}{\textsc{##1}}}
\def\HLSCz##1{\textcolor{black}{\textsc{##1}}}
%    \end{macrocode}
%
% Colored teletype writer text.
%    \begin{macrocode}
\def\HLTTa##1{\textcolor{blue}{\texttt{##1}}}
\def\HLTTb##1{\textcolor{cyan}{\texttt{##1}}}
\def\HLTTc##1{\textcolor{green}{\texttt{##1}}}
\def\HLTTd##1{\textcolor{magenta}{\texttt{##1}}}
\def\HLTTe##1{\textcolor{red}{\texttt{##1}}}
\def\HLTTf##1{\textcolor{yellow}{\texttt{##1}}}
\def\HLTTq##1{\textcolor{ForestGreen}{\texttt{##1}}}
\def\HLTTr##1{\textcolor{PineGreen}{\texttt{##1}}}
\def\HLTTz##1{\textcolor{black}{\texttt{##1}}}
%    \end{macrocode}
%
% Colored italic and teletype writer text.
%    \begin{macrocode}
\def\HLITTTa##1{\textcolor{blue}{\textsl{\texttt{##1}}}}
\def\HLITTTb##1{\textcolor{cyan}{\textsl{\texttt{##1}}}}
\def\HLITTTc##1{\textcolor{green}{\textsl{\texttt{##1}}}}
\def\HLITTTd##1{\textcolor{magenta}{\textsl{\texttt{##1}}}}
\def\HLITTTe##1{\textcolor{red}{\textsl{\texttt{##1}}}}
\def\HLITTTf##1{\textcolor{yellow}{\textsl{\texttt{##1}}}}
\def\HLITTTz##1{\textcolor{black}{\textsl{\texttt{##1}}}}
%    \end{macrocode}
%
% Black text inside a colored box.
%    \begin{macrocode}
\def\HLCBBa##1{\FvrbEx@ColoredBox{##1}{blue}{black}}
\def\HLCBBb##1{\FvrbEx@ColoredBox{##1}{cyan}{black}}
\def\HLCBBc##1{\FvrbEx@ColoredBox{##1}{green}{black}}
\def\HLCBBd##1{\FvrbEx@ColoredBox{##1}{magenta}{black}}
\def\HLCBBe##1{\FvrbEx@ColoredBox{##1}{red}{black}}
\def\HLCBBf##1{\FvrbEx@ColoredBox{##1}{yellow}{black}}
\def\HLCBBz##1{\FvrbEx@ColoredBox{##1}{black}{black}}
%    \end{macrocode}
%
% White text inside a colored box.
%    \begin{macrocode}
\def\HLCBWa##1{\FvrbEx@ColoredBox{##1}{blue}{white}}
\def\HLCBWb##1{\FvrbEx@ColoredBox{##1}{cyan}{white}}
\def\HLCBWc##1{\FvrbEx@ColoredBox{##1}{green}{white}}
\def\HLCBWd##1{\FvrbEx@ColoredBox{##1}{magenta}{white}}
\def\HLCBWe##1{\FvrbEx@ColoredBox{##1}{red}{white}}
\def\HLCBWf##1{\FvrbEx@ColoredBox{##1}{yellow}{white}}
\def\HLCBWz##1{\FvrbEx@ColoredBox{##1}{black}{white}}
%    \end{macrocode}
%
% Colored underlined text.
%    \begin{macrocode}
\def\HLSa##1{\color{blue}\underline{##1}}
\def\HLSb##1{\color{cyan}\underline{##1}}
\def\HLSc##1{\color{green}\underline{##1}}
\def\HLSd##1{\color{magenta}\underline{##1}}
\def\HLSe##1{\color{red}\underline{##1}}
\def\HLSf##1{\color{yellow}\underline{##1}}
\def\HLSz##1{\color{black}\underline{##1}}
%    \end{macrocode}
%
% Colored underlined colored text (with the same color).
%    \begin{macrocode}
\def\HLSaa##1{\FvrbEx@ColoredUnderline{blue}{black}{##1}}
\def\HLSbb##1{\FvrbEx@ColoredUnderline{cyan}{black}{##1}}
\def\HLScc##1{\FvrbEx@ColoredUnderline{green}{black}{##1}}
\def\HLSdd##1{\FvrbEx@ColoredUnderline{magenta}{black}{##1}}
\def\HLSee##1{\FvrbEx@ColoredUnderline{red}{black}{##1}}
\def\HLSef##1{\FvrbEx@ColoredUnderline{yellow}{black}{##1}}
\def\HLSez##1{\FvrbEx@ColoredUnderline{black}{black}{##1}}
%    \end{macrocode}
%
% End of \cs{Highlight@Attributes}.
%    \begin{macrocode}
}
%    \end{macrocode}
%
% \begin{macro}{\NoHighlight@Attributes}
%     \cs{NoHighlight@Attributes} is an internal macro to inhibit all
%   the highlighting macros define by \cs{Highlight@Attributes}. It is
%   necessary  to call it before to insert the formatted part, as highlighting
%   process must concern only the verbatim one.
% \end{macro}
%
%    \begin{macrocode}
\def\NoHighlight@Attributes{%
%    \end{macrocode}
%
% First, we re-establish the active catcodes for the verbatim mode.
%    \begin{macrocode}
\catcode`\^^a7=0\relax%
\catcode`\^^b5=1\relax%
\catcode`\^^b6=2\relax%
%    \end{macrocode}
%
% Desactivation of the highlighting macros.
%    \begin{macrocode}
\def\HLa##1{##1}%
\def\HLb##1{##1}%
\def\HLc##1{##1}%
\def\HLd##1{##1}%
\def\HLe##1{##1}%
\def\HLf##1{##1}%
\def\HLBFa##1{##1}%
\def\HLBFb##1{##1}%
\def\HLBFc##1{##1}%
\def\HLBFd##1{##1}%
\def\HLBFe##1{##1}%
\def\HLBFf##1{##1}%
\def\HLITa##1{##1}%
\def\HLITb##1{##1}%
\def\HLITc##1{##1}%
\def\HLITd##1{##1}%
\def\HLITe##1{##1}%
\def\HLITf##1{##1}%
\def\HLCBBa##1{##1}%
\def\HLCBBb##1{##1}%
\def\HLCBBc##1{##1}%
\def\HLCBBd##1{##1}%
\def\HLCBBe##1{##1}%
\def\HLCBBf##1{##1}%
\def\HLCBBz##1{##1}%
\def\HLCBWa##1{##1}%
\def\HLCBWb##1{##1}%
\def\HLCBWc##1{##1}%
\def\HLCBWd##1{##1}%
\def\HLCBWe##1{##1}%
\def\HLCBWf##1{##1}%
\def\HLCBWz##1{##1}%
%    \end{macrocode}
%
% End of \cs{NoHighlight@Attributes}.
%    \begin{macrocode}
}
%    \end{macrocode}
%
% Activation of the highlighting macros.
%    \begin{macrocode}
\Highlight@Attributes
%    \end{macrocode}
%
%</hcolor>
%
% \section{Test file}
%
%<*t-fvrbex>
% \iffalse meta-comment, etc.
%% File `t-fvrbex'
%%
% \fi
%
%    \begin{macrocode}
\documentclass{article}

\usepackage[bawcolor,pstricks]{fvrb-ex}
\pstrickstrue
\usepackage[T1]{fontenc}
\usepackage[latin1]{inputenc}
\usepackage[charter]{mathdesign}
\usepackage{url}
\usepackage{xcolor}

\begin{document}

\title{Test file for the `\textsf{fvrb-ex}' package}
\author{Denis Girou\\CNRS/IDRIS\\Orsay -- France\\
        {\footnotesize email: Denis.Girou@idris.fr}}
\date{Version 1.2\\March 27, 1998}

\maketitle

\RecustomVerbatimEnvironment{Verbatim}{Verbatim}
  {gobble=2,commentchar=^^a3,numbers=left,numbersep=3pt,frame=single}

\section{\texttt{Example} environment}

\begin{Verbatim}
  \begin{Example}
    First verbatim line.
    Second verbatim line.
    Third verbatim line.
  \end{Example}
\end{Verbatim}

\begin{Example}
  First verbatim line.
  Second verbatim line.
  Third verbatim line.
\end{Example}

\begin{Verbatim}
  \begin{Example}[frame=lines,framerule=1mm,numbers=left]
    First verbatim line.
    Second verbatim line.
    Third verbatim line.
  \end{Example}
\end{Verbatim}

\begin{Example}[frame=lines,framerule=1mm,numbers=left]
  First verbatim line.
  Second verbatim line.
  Third verbatim line.
\end{Example}

\section{\texttt{CenterExample} environment}

\begin{Verbatim}
  \begin{CenterExample}[frame=single,numbers=right]
    First verbatim line.
    Second verbatim line.
    Third verbatim line.
  \end{CenterExample}
\end{Verbatim}

\begin{CenterExample}[frame=single,numbers=right]
  First verbatim line.
  Second verbatim line.
  Third verbatim line.
\end{CenterExample}

\begin{Verbatim}
  \begin{CenterExample}[frame=lines,numbers=left]
    ^^a7HLa^^b5First^^b6 verbatim line.
    ^^a7HLb^^b5Second^^b6 verbatim line.
    ^^a7HLCBWz^^b5Third^^b6 verbatim line.
  \end{CenterExample}
\end{Verbatim}

\begin{CenterExample}[frame=lines,numbers=left]
  ^^a7HLa^^b5First^^b6 verbatim line.
  ^^a7HLb^^b5Second^^b6 verbatim line.
  ^^a7HLCBWz^^b5Third^^b6 verbatim line.
\end{CenterExample}

\section{\texttt{SideBySideExample} environment}

\begin{Verbatim}
  \begin{SideBySideExample}[xrightmargin=5cm,frame=lines,
                            numbers=left]
    First verbatim line.
    Second verbatim line.
    Third verbatim line.
  \end{SideBySideExample}
\end{Verbatim}

\begin{SideBySideExample}[xrightmargin=5cm,frame=single,numbers=left]
  First verbatim line.
  Second verbatim line.
  Third verbatim line.
\end{SideBySideExample}

\ifpstricks                     % If PSTricks is available

\section{\texttt{PCenterExample} environment}

\begin{Verbatim}
  \fvset{frame=lines,framerule=0.5mm,numbers=left}

  \begin{PCenterExample}(-0.5,-0.5)(0.5,0.5)
    \setlength{\unitlength}{1cm}
    \put(0,0){\circle{1}}
  \end{PCenterExample}

  \showgrid
  \begin{PCenterExample}(-1,-1)(1,1)
    \setlength{\unitlength}{1cm}
    \put(0,0){\circle{1}}
  \end{PCenterExample}
\end{Verbatim}

{\fvset{frame=lines,framerule=0.5mm,numbers=left}
\begin{PCenterExample}(-0.5,-0.5)(0.5,0.5)
  \setlength{\unitlength}{1cm}
  \put(0,0){\circle{1}}
\end{PCenterExample}
\showgrid
\begin{PCenterExample}(-1,-1)(1,1)
   \setlength{\unitlength}{1cm}
   \put(0,0){\circle{1}}
\end{PCenterExample}
}

\section{\texttt{PSideBySideExample} environment}

\begin{Verbatim}
  \fvset{frame=single,xrightmargin=5cm}
  \begin{PSideBySideExample}(-2,-1)(2,1)
    \psellipse*[linecolor=yellow](2,1)
  \end{PSideBySideExample}
  \showgrid
  \begin{PSideBySideExample}(-2,-1)(2,1)
    \psellipse[linestyle=dashed](2,1)
  \end{PSideBySideExample}
\end{Verbatim}

{\fvset{frame=single,xrightmargin=5cm}
\begin{PSideBySideExample}(-2,-1)(2,1)
  \psellipse*[linecolor=yellow](2,1)
\end{PSideBySideExample}

\showgrid
\begin{PSideBySideExample}(-2,-1)(2,1)
  \psellipse[linestyle=dashed](2,1)
\end{PSideBySideExample}
}

\begin{Verbatim}
  \fvset{frame=single,xrightmargin=5cm}
  \begin{PSideBySideExample}(-2,-1)(2,1)
    \psellipse[linestyle=^^a7HLCBWe^^b5dashed^^b6](2,1)
  \end{PSideBySideExample}
  \begin{PSideBySideExample}[numbers=right](-2,-1)(2,1)
    \psellipse[linestyle=^^a7HLe^^b5dotted^^b6](2,1)
  \end{PSideBySideExample}
\end{Verbatim}

{\fvset{frame=single,xrightmargin=5cm}
\begin{PSideBySideExample}(-2,-1)(2,1)
  \psellipse[linestyle=^^a7HLCBWe^^b5dashed^^b6](2,1)
\end{PSideBySideExample}
\begin{PSideBySideExample}[numbers=right](-2,-1)(2,1)
  \psellipse[linestyle=^^a7HLe^^b5dotted^^b6](2,1)
\end{PSideBySideExample}


\else                           % If PSTricks is not available
\begin{quote}
  \section{\texttt{PCenterExample} and \texttt{PSideBySideExample}
environments}

  \textbf{\large Warning!} These two environments are not demonstrated here,
because PSTricks was not found on this platform.
\end{quote}
\fi

\end{document}
%    \end{macrocode}
%
%</t-fvrbex>
%
% \Finale
% \PrintIndex
% \PrintChanges
%
\endinput
%% 
%% End of file `fvrb-ex.dtx'
